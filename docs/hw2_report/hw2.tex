\documentclass[10pt, letterpaper]{article}
\usepackage[utf8]{inputenc}
\usepackage{amsmath}
\usepackage{amssymb}
\usepackage{bbm}
\usepackage{booktabs}
\usepackage{caption}
\usepackage{color}
\usepackage[shortlabels]{enumitem}
\usepackage{fancyhdr}
\usepackage{hyperref}
\usepackage{geometry}
\geometry{a4paper,scale=0.8}
\usepackage{graphicx}
\graphicspath{ {./img/}}
\usepackage{listings}
\usepackage{mathtools}
\usepackage{mathrsfs}
\usepackage{setspace}
\renewcommand{\baselinestretch}{1.3}

% set-up header & footer
\pagestyle{empty}
\fancyhf{}
\cfoot{\thepage}
\lhead{%
\textbf{University of California, Berkeley} \\
Department of Civil \& Environ. Eng.
}
\rhead{\textbf{CS 285 Deep Reinforcement Learning}\\\date{\today}}

\title{%
    \textbf{Homework 2}
}
\author{Juanwu Lu (3037432593)\\ \small(M.Sc. Civil Engineering, UC Berkeley)}
\date{}

% set-up code listing
\definecolor{dkgreen}{rgb}{0,0.6,0}
\definecolor{gray}{rgb}{0.5,0.5,0.5}
\definecolor{manuve}{rgb}{0.58,0,0.82}

\lstset{frame=tb,
    language=Python,
    aboveskip=3mm,
    belowskip=3mm,
    showstringspaces=false,
    columns=flexible,
    basicstyle={\small\ttfamily},
    numbers=none,
    numberstyle=\tiny\color{gray},
    keywordstyle=\color{blue},
    commentstyle=\color{dkgreen},
    stringstyle=\color{manuve},
    breaklines=true,
    breakatwhitespace=true,
    tabsize=3
}

\begin{document}
    \maketitle
    \captionsetup[figure]{labelfont={bf},labelformat={default},labelsep=period,name={Figure}}
    \captionsetup[table]{labelfont={bf},labelformat={default},labelsep=period,name={TABLE}}
    \thispagestyle{fancy}
    \pagestyle{plain}

    % Experiment 1
    \section*{Experiment 1 CartPole}
    \textbf{Answers:}
    \begin{itemize}
        \item The \textbf{reward-to-go} estimator has a better performance without advantage-standardization. Compare the green with orange curves in both figure 1(a) and (b), reward-to-go value estimators converge faster and are more stable across the training process.
        \item From my experiment results, advantage standardization helps in small batch experiments, but does not in large batch experiments.
        \item From my experiment results, batch size did make an impact to the training, with a larger batch size helps stablize the training.
    \end{itemize}

    \begin{figure}[thbp]
        \centering
        \includegraphics[width=\textwidth]{exp_01.png}
        \caption{Visualization of learning curves for (a) small batch experiments and (b) large batch experiments.}
        \label{fig:1}
    \end{figure}

    \textbf{Command-line Codes}
    \begin{lstlisting}
        echo 'Running small batch w/o reward_to_go w/ standardized_advatages';
        python $1 --env_name CartPole-v0 -n 100 -b 1000 -dsa --exp_name q1_sb_no_rtg_dsa;

        echo 'Running small batch w/ reward_to_go w/ standardized_advatages'; 
        python $1 --env_name CartPole-v0 -n 100 -b 1000 -rtg -dsa --exp_name q1_sb_rtg_dsa;

        echo 'Running small batch w/ reward_to_go w/o standardized_advatages';
        python $1 --env_name CartPole-v0 -n 100 -b 1000 -rtg --exp_name q1_sb_rtg_na;

        echo 'Running large batch w/o reward_to_go w/ standardized_advatages';
        python $1 --env_name CartPole-v0 -n 100 -b 5000 -dsa --exp_name q1_lb_no_rtg_dsa;

        echo 'Running large batch w/ reward_to_go w/ standardized_advatages'; 
        python $1 --env_name CartPole-v0 -n 100 -b 5000 -rtg -dsa --exp_name q1_lb_rtg_dsa;

        echo 'Running large batch w/ reward_to_go w/o standardized_advatages';
        python $1 --env_name CartPole-v0 -n 100 -b 5000 -rtg --exp_name q1_lb_rtg_na;
    \end{lstlisting}

    \section*{Experiment 2 InvertedPendulum}

    \textbf{Answers:}
    
    From my experiments, the optimal setting combination is \texttt{b*=500} and \texttt{r*=0.01}. Using this setting, I obtain a learning curve as shown in figure 2. Although this settings reaches a best score 1000 the fastest, the average return is unstable and shows occasion extreme decays.
    
    \begin{figure}[thbp]
        \centering
        \includegraphics[width=0.6\textwidth]{exp_02.png}
        \caption{Learning curve with optimal settings.}
        \label{fig:2}
    \end{figure}

    \textbf{Command-line Codes}
    \begin{lstlisting}
        echo "Searching for optimal batch and learning rate...";
        for BATCH in 500 1000 2500 5000 7500 
        do
            for LR in 0.005 0.001 0.005 0.01 0.05
            do
                echo "Now running on batch_size=${BATCH}, learning rate=${LR}."
                NAME="q2_b${BATCH}_r${LR}";
                python $1 --env_name InvertedPendulum-v4 --ep_len 1000 --discount 0.9 -n 100 -l 2 -s 64 -b $BATCH -lr $LR -rtg --exp_name $NAME;
            done
        done
    \end{lstlisting}

    \newpage
    \section*{Experiment 3 LunarLander}

    \begin{figure}[thbp]
        \centering
        \includegraphics[width=\textwidth]{exp_03.png}
        \caption{Learning curves for the \texttt{LunarLander-v2}.}
        \label{fig:3}
    \end{figure}

    \textbf{Command-line Codes}
    \begin{lstlisting}
        echo "Running LunarLander with reward-to-go estimator.";
        python cs285/scripts/run_hw2.py --env_name LunarLanderContinuous-v2 --ep_len 1000 --discount 0.99 -n 100 -l 2 -s 64 -b 40000 -lr 0.005 --reward_to_go --nn_baseline --exp_name q3_b40000_r0.005;
    \end{lstlisting}

    \newpage
    \section*{Experiment 4 HalfCheetah}

    \textbf{Answer:}

    The results from trying out different settings are shown in the following figure 4. From my experiments, the optimal settings are \texttt{b*=30000} and \texttt{lr*=0.02}.

    \begin{figure}[h!]
        \centering
        \includegraphics[width=0.75\textwidth]{exp_04_1.png}
        \caption{Learning curves for the \texttt{HalfCheetah} experiments of different settings.}
        \label{fig:4}
    \end{figure}

    Using the optimal settings to investigate the effect of reward-to-go and value estimator network, the results are shown in the following figure 5.
    The switching from vanilla policy gradients to a reward-to-go formulation has more significant acceleration to the training converge than the value estimator.

    \begin{figure}[h!]
        \centering
        \includegraphics[width=0.75\textwidth]{exp_04_2.png} 
        \caption{Learning curves of different policy gradient settings under optimal batch size and learning rate.}
        \label{fig:5}
    \end{figure}

    \textbf{Command-line Codes}
    \begin{lstlisting}
        echo "Search for optimal batch_size and learning rate for HalfCheetah.";
        for BATCH in 10000 30000 50000
        do
            for LR in 0.005 0.01 0.02
            do
                echo "Now running on batch_size=${BATCH}, learning_rate=${LR}.";
                NAME="q4_search_b${BATCH}_lr${LR}_rtg_nnbaseline";
                python $1 --env_name HalfCheetah-v4 --ep_len 150 --discount 0.95 -n 100 -l 2 -s 32 -b $BATCH -lr $LR -rtg --nn_baseline --exp_name $NAME;
            done
        done


        BATCH=$2
        LR=$3

        echo "Running with optimal batch_size and learning rate for HalfCheetah.";
        echo "Running baseline";
        python $1 --env_name HalfCheetah-v4 --ep_len 150 --discount 0.95 -n 100 -l 2 -s 32 -b $BATCH -lr $LR --exp_name q4_b${BATCH}_r${LR};
        echo "Running reward-to-go";
        python $1 --env_name HalfCheetah-v4 --ep_len 150 --discount 0.95 -n 100 -l 2 -s 32 -b $BATCH -lr $LR -rtg --exp_name q4_b${BATCH}_r${LR}_rtg;
        echo "Running with baseline neural network";
        python $1 --env_name HalfCheetah-v4 --ep_len 150 --discount 0.95 -n 100 -l 2 -s 32 -b $BATCH -lr $LR --nn_baseline --exp_name q4_b${BATCH}_r${LR}_nnbaseline;
        echo "Running with all";
        python $1 --env_name HalfCheetah-v4 --ep_len 150 --discount 0.95 -n 100 -l 2 -s 32 -b $BATCH -lr $LR -rtg --nn_baseline --exp_name q4_b${BATCH}_r${LR}_rtg_nnbaseline
    \end{lstlisting}

    \newpage
    \section*{Experiment 5 HopperV4}

    \textbf{Answers:}

    The averaged evaluation returns with respect to training steps given different $\lambda$ settings is as shown in figure 6. Results show that the evaluation performance at the same timestep increases with a $\lambda$ increasing from 0.00 to 1.00, meaning reducing bias has benefits on the trainig procedure.

    \begin{figure}[thbp]
       \centering
       \includegraphics[width=0.9\textwidth]{exp_05.png} 
       \caption{Learning curves for the \texttt{Hopper-v4} with different $\lambda$ settings.}
       \label{fig:6}
    \end{figure}

    \textbf{Command-line Codes}
    \begin{lstlisting}
        echo "Search for the optimal GAE lambda setting...";
        for LAMBDA in 0.00 0.95 0.98 0.99 1.00
        do
            echo "Now running on gae_lambda = ${LAMBDA}.";
            NAME="q5_b2000_r0.001_lambda${LAMBDA}";
            python $1 --env_name Hopper-v4 --ep_len 1000 --discount 0.99 -n 300 -l 2 -s 32 -b 2000 -lr 0.001 --reward_to_go --nn_baseline --action_noise_std 0.5 --gae_lambda $LAMBDA --exp_name $NAME;
        done
    \end{lstlisting}
    
\end{document}